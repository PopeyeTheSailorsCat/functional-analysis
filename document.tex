\documentclass{article}
\usepackage[utf8]{inputenc}
\usepackage[a4paper, left=2.5cm, right=1.5cm, top=2.5cm, bottom=2.5cm]{geometry} % настройки полей документа
\usepackage[utf8]{inputenc}
\usepackage[english,russian]{babel}
\usepackage{amsmath}
\usepackage{tikz,amstext}
\usepackage{graphicx}
\usepackage{amssymb}
\usepackage{lscape}
\usepackage{indentfirst}

\newlength{\tempheight}
\graphicspath{{/.graph/}}

\newcommand{\Let}[0]{%
	\mathbin{\text{\settoheight{\tempheight}{\mathstrut}\raisebox{0.5\pgflinewidth}{%
				\tikz[baseline,line cap=round,line join=round] \draw (0,0) --++ (0.4em,0) --++ (0,1.5ex) --++ (-0.4em,0);%
}}}}


\begin{document}
	\titlepage
	\large{}
	
	\title{Функциональный анализ}
	\author{Мамаева Анастасия, Веденичев Дмитрий}
	\date{Осень - Зима 2020}
	
	\maketitle
	
	\section{Счетные множества}
	
	\noindent Условимся обозначать множества большими буквами латинского алфавита. Пусть есть некоторое множество  X такое, что оно содержит любое другое $  A  \subset  X$. Элементы этих множеств будем обозначать маленькими буквами $ x, y \in A$.
	\hfill \break
	
	\noindentКроме того, введём понятие семейства множеств
	\noindent$\{ A_\alpha\}_\alpha \in\Lambda$, где $\Lambda$ - множество индексов, $\alpha$ - индекс.
	
	\hfill \break
	\noindent{\it Объединением множеств} называется
	$\bigcup \limits_{\alpha \in \Lambda} A_\alpha = \{x \in X \vert \: \exists \: \alpha_o \in \Lambda \: \colon \: x \in A_{\alpha_o} \:$$\}$
	
	\hfill \break
	\noindent{\it Пересечение множеств} $\bigcap\limits_{\alpha \in \Lambda}A_\alpha = \{x \in X \vert \:x \in A_\alpha, \forall \alpha \in \Lambda \}$
	
	\hfill \break
	\noindent{\it Разность двух множеств } $A \setminus B = \{x \in X \vert \:x \in A, \: x \notin A \} $
	
	\hfill \break
	\noindent{\it Дополнение } $cA = X \setminus A = \{x \in X \vert \: x \notin A \}$
	
	\hfill \break
	\noindentУтверждения $\:\:1^0 \: c(\bigcup\limits_{\alpha \in \Lambda}A_\alpha) = \bigcap\limits_{\alpha \in \Lambda}(cA_\alpha)$
	$\:\:\:\:\:\:\:\:2^0 \: c(\bigcap\limits_{\alpha \in \Lambda}A_\alpha) = \bigcup\limits_{\alpha \in \Lambda}(cA_\alpha)$
	
	\hfill \break
	\textbf{Доказательство 1.}
	\hfill \break
	\begin{equation*}
		\begin{aligned}
			&\Let  x \in c\left( \bigcup_{\alpha\in\Lambda} A_\alpha\right) \Leftrightarrow x\notin \bigcup_{\alpha\in\Lambda} A_\alpha \Leftrightarrow \nexists \alpha_0 \in \Lambda : x\in A_{\alpha_0}
			\Leftrightarrow x\notin A_\alpha,\forall \alpha \in \Lambda \Leftrightarrow\\
			&\Leftrightarrow x \in cA_\alpha,\forall\alpha\in\Lambda
			\Leftrightarrow x\in\bigcap_{\alpha \in \Lambda} \left(cA_\alpha\right)
			\Box
		\end{aligned}
	\end{equation*}\par
	
	\hfill \break
	\textbf{Доказательство 2.}
	\hfill \break
	\begin{equation*}
		\begin{aligned}
			&\Let x \in c\left(\bigcap_{\alpha\in\Lambda} A_\alpha\right) \Leftrightarrow
			x \notin \bigcap_{\alpha\in\Lambda} A_\alpha \Leftrightarrow
			\exists \alpha_0 \in\Lambda : x \in A_{\alpha_0} \Leftrightarrow
			x \in cA_{\alpha_0} \Leftrightarrow
			x \in \bigcup_{\alpha \in \Lambda} \left(cA_\alpha\right)
			\\
			&\text{Поскольку }
			\exists cA_{\alpha_0} : x\in  cA_{\alpha_0}
			\Box
		\end{aligned}			 
	\end{equation*}\par
	
	
\end{document}